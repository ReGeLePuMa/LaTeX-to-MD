% Petrea Andrei 331CC
\documentclass[a4paper, 12pt, twoside]{article}
\newcommand{\geometrysetting}{vmargin = 2cm, hmargin = 2cm}\newcommand{\languagesetting}{romanian}
\usepackage{hyperref}

\title{Document Exemplu 1}
\date{13.10.2023}

\begin{document}
\maketitle

\section{Introducere}\label{sec:intro}

Acesta este un document LaTeX de exemplu pe care îl vom transforma într-un document Markdown.

\textbf{Text cu Bold.}

\emph{Text Emphasized.}

\begin{quotation}
    Alegerile parlamentare din România din 1946 au fost convocate pe 19 noiembrie 1946 în Regatul României. Rezultatele oficiale i-au dat câștigători pe comuniștii români (PCR) și pe aliații lor din Blocul Partidelor Democrate (BPD):
Uniunea Populară Maghiară (UPM/MNSz), facțiunea proguvernamentală țărănistă din jurul lui Dr. Nicolae Lupu și Comitetul Democrat Evreiesc.
Aceste alegeri au marcat un pas decisiv spre destabilizarea monarhiei constituționale române și instaurarea regimului comunist în țară la sfârșitul anului următor.
\end{quotation}

\begin{enumerate}
    \item Primul element
    \item Al doilea element
    \item Al treilea element 
        \begin{enumerate}
            \item Primul sub-element
            \item Al doilea sub-element
        \end{enumerate}
\end{enumerate}

\begin{verbatim}
#include <stdio.h>

int main() {
    printf("Hello, World!\n");
    return 0;
}
\end{verbatim}

\end{document}