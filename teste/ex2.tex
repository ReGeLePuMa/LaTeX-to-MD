% Petrea Andrei 331CC
% This is an example LaTeX file. The percent sign is used to mark the
% start of a comment.
%
% - Michael Weeks, January, 2003
%
\documentclass{article}
\usepackage[dvips]{graphics}
\begin{document}
\title{My Example LaTeX Paper}
\author{Michael Weeks}
\maketitle

\textheight{7.62in}
% The above command controls the height of the text

\begin{abstract}
This is the abstract. You can use this file to start your own LaTeX file,
and just delete the stuff you do not need. \LaTeX is a lot like working
with HTML: you can specify where text effects begin, and where they end.
\end{abstract}

\section{Introduction}\label{sec:intro}
Here is the introduction.
Since there is no blank line between these first 3 sentences, they are
treated as one paragraph.
Here is a vertical space (of 0.3 inches):
\vspace{.3in}

And here is a \hspace{.3in}horizontal space (of 0.3 inches).

A blank line means that the last paragraph is over, and it is time to start
a new one.

You can have text in \textit{italics} font, or in \textbf{bold} font and 
text \underline{underlined}. You can also have \emph{emphasized} text.


Citing a reference: This is a book about VLSI \cite{Weste93}.
Also, the references contain a good conference paper \cite{LiY88},
and a good journal article \cite{BiS92}.

What if you want to include a figure?
Here is an example, figure~\ref{fig:phasor1}, that is saved in
encapsulated postscript format.

\begin{figure}
  \centering
  \caption{A complex number can be shown as a point or a vector}
  \label{fig:phasor1}
\end{figure}


Skip a lot of space \bigskip vertically.

\section{Here is some Math}\label{sec:math}
This is different from the previous section, section~\ref{sec:intro}.
This section gives some examples of Math.

Using superscript: 2^3.

Using subscript: x_0.

If you use a character, but LaTeX complains about it, try putting a
back-slash before it. For example,
f = x^y uses the carat character.
If you want to end a line, use 2 back-slashes.
If you want the backslash character \\ in your document,
this can be done, too.

For more info, see


\href{https://www.maths.tcd.ie/~7ddwilkins/LaTeXPrimer}{LaTeXPrimer}


% Now here is the reference section.

\begin{thebibliography}{99}

% Book
\emph{Weste93} Neil H. E. Weste and Kamran Eshraghian, \textit{Principles
of CMOS VLSI Design}, 2nd ed. Reading, MA: Addison-Wesley, 1993.

%Example of a Conference Paper
\emph{LiY88} R. A. Lincoln and K. Yao, ``Efficient Systolic Kalman
Filtering Design by Dependence Graph Mapping,'' in \textit{VLSI Signal
Processing, III}, IEEE Press, R. W. Brodersen and H. S. Moscovitz Eds.,
1988, pp.~396--410.

% Example of a Journal Paper
\emph{BiS92} C. H. Bischof and G. M. Shroff, ``On Updating Signal
Subspaces,'' \textit{IEEE Trans. on Signal Processing}, vol.~40, no.~1,
pp.~96--105, Jan. 1992.

\end{thebibliography}
\end{document}